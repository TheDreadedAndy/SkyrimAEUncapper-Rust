% Setup
\documentclass[12pt]{amsart}
\usepackage{array,amssymb,amsmath,amsthm,amsfonts,latexsym,euscript}
\usepackage[margin=0.7in]{geometry}
\usepackage{hyperref}
\raggedbottom
% Contents
\title{Skyrim Uncapper (SE/AE) User Guide}
\author{Andrew Spaulding (Kasplat)}
\begin{document}
\maketitle
\pagebreak

% Helper functions
\newcommand{\startblock}{\begin{minipage}{\textwidth}}
\newcommand{\stopblock}{\end{minipage}}

\section{Introduction}

In this file, I hope to clear up some of the confusion surrounding how to
configure this plugin. My hope is that this file will obviate the need
to ask me questions on the Nexus forum, but I will, of course, be willing
to answer any questions regardless.

\pagebreak
\section{Change Log}

TODO

\pagebreak
\section{Bug Reports}

If you encounter an issue with the plugin, which may include strange game
behavior, plugin incompatibility, or a CTD (which would hopefully include
a panic message box from the plugin), please report it to me on the BUGS
section of the nexus page. Please do not report it in the posts section, as
this section has less visiblity to me and other users (e.g., I do not get
notified on Nexus when someone makes a post in that section).

When reporting a bug, please include a copy of the log file found in
<Documents>/My Games/Skyrim Special Edition/SKSE/SkyrimUncapper.log,
as well as a brief description of the issue/what you were doing when you
encountered it. This information will allow me to resolve the problem promptly.

\pagebreak
\section{INI file generation and updating}

\startblock
\subsection{INI file location}

The location of the INI file will depend on a few factors. Those being the
current version of the plugin you are running, and whether or not you have,
at any point, run the game with a previous version installed. It will also
depond on if you are using a mod manager, though I will assume that everyone
is because if you aren't you need to rethink the choices in your life that
brought you to this point.

The path to the INI file is always /data/SKSE/plugins/SkyrimUncapper.ini.
Where the file actually gets place will depend on your plugin version and
mod manager.
\stopblock

\startblock
\subsubsection{Kassent/Vadfromnu's uncapper, or versions 1.1.0 and below, or for those updating from any of these}

Those who are using/used these versions of the plugin will have had their
INI file generated. MO2 will place this generated INI file in the overwrite
directory. Vortex should place the file in Skyrim's game directory. You will
need to run the game with the plugin at least once before configuring it.
Note that the default configuration uses all vanilla settings.
\stopblock

\startblock
\subsubsection{Versions 2.0.0+}

These versions package the INI file with the zip file. If using MO2, this file
will be ignored if there is also an INI file in the overwrite directory.
Assuming there is no INI file in the overwrite directory, the file will appear
in the same directory as the plugin itself.
\stopblock

\startblock
\subsection{Updating the INI file}

When new fields are added to the INI file by an update to the plugin, some
versions of the plugin will automatically add those fields. How each version
goes about this process varies, however.
\stopblock

\startblock
\subsubsection{Kassent/Vadfromnu's uncapper, or versions 1.1.0 and below}

These versions of the uncapper track the file version using the \textit{Version}
field in the \textit{[General]} section of the INI. If this field is outdated
(according to the value that said version of the plugin expects) the INI file
is regenerated with the currently stored configuration values as loaded from
the INI.

Note that this regeneration has a few issues. Any field which was not loaded
correctly will be lost, and all user-added comments/formatting are stripped
from the INI file in this process. It is as if the INI file was generated again
from scratch, except the previously known values are used instead of the
default values.
\stopblock

\startblock
\subsubsection{Versions 2.0.0 to 2.1.5}

These versions of the uncapper make no attempts to automatically update the INI
file. These versions will issue warnings in the log file when they detect
that a field/section is missing from the INI file, however any actual updates
to the file must be manually applied by the user.
\stopblock

\startblock
\subsubsection{Versions 2.2.0 and above}

These versions will automatically append missing fields to the end of the
section they should appear in, and missing sections to the end of the file.
The formatting and comments within the INI files should not be altered by this
process (or should only be minimally altered). A message will be displayed to
the user when the game is launched if their INI file was updated.

Note that this process happens independent of the values of any field. This
means that the \textit{Version} field used in previous versions of the uncapper
is ignored by these versions.
\stopblock

\pagebreak
\section{INI Configuration}

The INI configuration is split into sections, with most sections having some
way to disable them entirely from the general section. This ability was
introduced in version 1.1.0 of the plugin.

\startblock
\subsection{[General]}

This section includes options to disable individual patches, as well as some
now depreciated options related to updating the INI file.
\stopblock

\startblock
\subsubsection{Author = <String>}

This field is ignored by all versions of the plugin.
\stopblock

\startblock
\subsubsection{Version = <unsigned int>}

This field is used by versions 1.1.0 and below to determine if the INI file
should be updated. Versions 2.0.0 and above ignore this field.
\stopblock

\subsubsection{bUseSkillCaps = <true/false>}

\subsubsection{bUseSkillFormulaCaps = <true/false>}

\subsubsection{bUseSkillFormulaCapsUIFix = <true/false>}

\subsubsection{bUseEnchanterCaps = <true/false>}

\subsubsection{bUseSkillExpGainMults = <true/false>}

\subsubsection{bUsePCLevelSkillExpMults = <true/false>}

\subsubsection{bUsePerksAtLevelUp = <true/false>}

\subsubsection{bUseAttributesAtLevelUp = <true/false>}

\subsubsection{bUseLegendarySettings = <true/false>}

\subsection{[SkillCaps]}

\pagebreak
\section{External Links}

\href{https://github.com/TheDreadedAndy/SkyrimAEUncapper-Rust}{GitHub}

\href{https://www.nexusmods.com/skyrimspecialedition/mods/82558}{Nexus}

\end{document}
